\begin{abstract}
The annual influenza outbreak leads to significant public health and economic 
burdens making it desirable to have prompt and accurate probabilistic forecasts 
of the disease spread. The United States Centers for Disease Control and 
Prevention (CDC) hosts annually a national flu forecasting competition which has 
led to the development of a variety of flu forecast modeling methods. For the 
first several years of the competition, the target to be forecast was weekly 
percentage of patients with an influenza-like illness (ILI), but in 2021 the 
target was changed to weekly hospitalization counts. Reliable state and national 
hospitalization data has only been available since 2021, but for ILI the data 
has been available since 2010 and has been successfully forecast for several 
seasons.
In this manuscript, we introduce a two component modeling framework for 
forecasting weekly hospitalizations utilizing both hospitalization data and ILI 
data. The first component is for modeling ILI data using a dynamic nonlinear 
Bayesian hierarchical model. The second component is for modeling 
hospitalizations as a function of ILI. For hospitalization forecasts, ILI is 
first forecasted and then hospitalizations are forecast with ILI forecasts used 
as a linear or quadratic predictor. In a simulation study, two ILI forecast 
models, including one similar to the winning model for two seasons of the CDC 
forecast competition from Osthus et al.
% \cite[]{osthus2019dynamic} 
% 
and a nonlinear Bayesian hierarchical model from Ulloa 
% \cite[]{ulloa2019}
are compared. Also assessed is the usefulness of including a systematic model 
discrepancy term in the ILI model. Forecasts of state and national 
hospitalizations for the 2023-24 flu season are made, and different modeling 
decisions are compared. We found that including a discrepancy component in the 
ILI model tends to improve forecasts during certain weeks of the year. We also 
found that other modeling decisions such as the exact nonlinear function to be 
used in the ILI model or the error distribution for hospitalization models may 
or may not be better than other decisions, depending on the season, location, 
or week of the forecast.
\end{abstract}
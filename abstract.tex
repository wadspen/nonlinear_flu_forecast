\begin{abstract}
The annual influenza outbreak leads to significant public health and economic 
burdens making it desirable to have prompt and accurate probabilistic forecasts 
of the disease spread. The United States Centers for Disease Control and 
Prevention hosts annually a national flu forecasting competition which has 
led to the development of a variety of flu forecast modeling methods. For the 
first several years of the competition, the target to be forecast was weekly 
percentage of patients with an influenza-like illness (ILI), but in 2021 the 
target was changed to weekly hospitalization counts. Reliable state and national 
hospitalization data has only been available since 2021, but for ILI the data 
has been available since 2010 and has been successfully forecast for several 
seasons.
In this manuscript, we introduce a two component 
Bayesian modeling framework for 
forecasting weekly hospitalizations utilizing both hospitalization data and ILI 
data. The first component is for modeling ILI data using a dynamic nonlinear 
Bayesian hierarchical model. This component includes modeling discrepancy 
between a functional model and the observed which requires thoughtful decisions
for modeling constraints and selecting prior parameter distributions.
The second component is for modeling 
hospitalizations as a function of ILI. For hospitalization forecasts, ILI is 
first forecasted and then hospitalizations forecasts are produced using
ILI forecasts 
as a linear or quadratic predictor. The combination of the models is done in a
manner similar to using the Bayesian cut.
In a simulation study, two ILI forecast 
models, including one similar to the winning model for two seasons of the CDC 
forecast competition from Osthus et al.
% \cite[]{osthus2019dynamic} 
% 
and a nonlinear Bayesian hierarchical model from Ulloa 
% \cite[]{ulloa2019}
are compared with other standard forecasts models.
The usefulness of including a systematic model 
discrepancy term in the ILI model is specifically addressed and proves 
to greatly improve forecasts. In a real data analysis, 
forecasts of state and national 
hospitalizations for the 2023-24 flu season are made and compared with forecasts
from 20 other competing models. The forecasts from the proposed methodology 
outperformed all but one of 20 competing models. 
\end{abstract}
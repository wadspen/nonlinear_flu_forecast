\documentclass{article}

\usepackage{enumerate}
\usepackage{hyperref}


\usepackage{xcolor}
\newcommand{\jarad}[1]{{\color{red} Jarad: #1}}
\newcommand{\spencer}[1]{{\color{green} Spencer: #1}}

\begin{document}
  
  
\section*{Review BA2412-010}
\subsection*{Overall assessment:}
This paper by Wadsworth and Niemi adapts two models originally designed to forecast outpatient
ILI data in order to apply them to influenza hospitalization data. This is motivated by a switch in the
target definition of the US CDC FluSight challenge. The paper is well-written and pleasant to read.
The overall idea of leveraging models from previous FluSight cycles for the new target is appealing.
However, several points require clarification. Given the focus of the journal, the Bayesian inference
aspect should be developed more thoroughly as it currently seems a little ad hoc. Moreover, it
currently does not become clear to which degree the proposed methods offer improved forecasts
compared to existing (and potentially much simpler) models. I provide some specific comments in
the following.


\subsection*{Specific comments:}

\begin{enumerate}[1.]
\item The entire Bayesian inference scheme seems somewhat heuristic.

\begin{enumerate}[a.]
\item The ILI model is run separately and then feeds into the hospitalization model. So there is no
flow of information from the hospitalization data to the ILI model, which there would be in a fully
Bayesian procedure. Is this done on purpose and if so what is the motivation? The proposed
procedure reminds me of the "Bayesian cut" in the version described by Plummer
\href{(https://link.springer.com/article/10.1007/s11222-014-9503-z)}{(https://link.springer.com/article/10.1007/s11222-014-9503-z)}.
This may be a useful reference.
\item The authors employ a "modular Bayesian approach" mixing maximum likelihood and
Bayesian estimation. The authors should provide more ample references to the literature concerning
this type of procedure (is "modular Bayesian" a standard term?).
\item Also, it should be argued explicitly why this approach was chosen over a standard fully
Bayesian approach in the setting at hand. Was a fully Bayesian approach also explored? Which
difficulties were encountered?

\spencer{Without assuming a value for the $\lambda$ parameter in the ASG model, there were
serious problems with parameter identifiability. 
Osthus et al. in their SIR model doe something similar by setting $S_0 = 0.9$}

\item Notably it would be helpful to comment on how this approach affects the uncertainty
propagation / overall quantification of forecast uncertainty.

\end{enumerate}


\item Several comments on the data and general modelling choices:

\begin{enumerate}[a.]
\item Influenza-like illness (ILI) is not a single pathogen, but a mix of several pathogens causing
similar symptoms. How reasonable is it to assume SIR dynamics for such an indicator? Do we
really assume that the tipping point due to susceptible depletion can be captured reasonably here? In
my understanding this would be the only reason to include an S compartment.

\spencer{I agree with these comments on the SIR model. Perhaps we emphasized the SIR model more than we 
should have. Osthus et al. use the SIR model in combination with a data driven model
in an attempt to capture some aspect of the science in their forecast model.
More important in the argument of this paper than that the specific SIR model is
a superior in some way for modeling ILI is that the ASG model may
be a reasonable alternative to existing compartmental models at least in terms
of forecast performance.}

\item My impression is that the SIR does not really have any mechanistic meaning here and is just a
way of parameterizing curves that look qualitatively like ILI seasons. The ASG approach can be
seen as an alternative parameterization for such curves. Could the authors comment on what
practical differences exist between the two parameterizations?

\spencer{Practially, the reason to use the SIR model is generally because it is meant
to capture some of the scientific understanding of a disease outbreak. Again, we argue 
that the ASG is a reasonable alternative to SIR models in the case of the seasonal flu
outbreak. The ASG shape with a rise to a peak and a subsequent fall is similar in shape
to the regular ILI trajectory. It has also been used successfully in forecasting of 
other pheneomena with similar rises and falls (i.e. crop moisture)}.



\item Is the hospitalization indicator specific to influenza? My understanding is that it is, but this
should be more clear in Section 2.2. If it is influenza-specific: to which degree should we expect ILI
to be a helpful predictor given that it also contains COVID, RSV etc.? The influenza dynamics are
likely somewhat hidden in the ILI indicator.
\spencer{Yes, hospitalizations here are specific to lab confirmed influenza patients. 
I believe figure 6 in the text gives reason to believe ILI is a good predictor. Maybe 
we should add some additional EDA with an $R^2$ value to further defend this?}


\item Does the hospitalization indicator refer to incident (new) hospitalization or prevalence, i.e.,
bed occupancy?
\spencer{The weekly hospitalization indicator refers to new incident hospitalizations}

\item Is there any "mechanistic" rationale behind the ARX formulation for the hospitalizations? If we
predict prevalence / bed occupancy I can sort of see why the differences between weeks would be
explained by ILI. If we predict new hospitalizations I am not sure what motivates this ARX
formulation.

\item Currently the ILI value from the same week is used in equation (7). Would there be a point in
exploring different lags?
\spencer{Exploring different lags is an important idea here, especially considering there
may be a reporting lag. However, the reporting lag is not as big an issue as it has
been in the past. We assessed ILI lags for both one and two weeks previous to the current
week and found that ILI had less predictive power. }

\item On page 14 it says: "all negative values of [...] were set to 0 to reflect realistic values of
hospitalizations". Could this be handled more naturally by appropriate distributional assumptions?

\end{enumerate}


\item And some remarks on the evaluation.

\begin{enumerate}[a.]

\item It would be interesting to have comparisons of performance beyond the variations of the two
proposed models. There should definitely be some naive baseline model (e.g., a persistence
forecast) and some out-of-the-box model applied directly to the hospitalizations (e.g. seasonal
ARIMA). Given the forecasting task was taken from the CDC challenge it would also be interesting
to see how the proposed models fare against the other participants of the challenge.

\item The measure the authors call LWIS tends to give a lot of weight to forecast tasks with low
(expected) incidence as the relative differences between predictions and observations can be quite
large then. The "regular" WIS gives most weight to weeks with high incidence. So in a sense both
are complementary and it may be helpful to also discuss regular WIS a little more.

\item Why was the CRPS used in the simulation and WIS in the data example? These two are sort of
the same (WIS approximates CRPS), so things would be more coherent if one was used throughout.
\spencer{Probably use CRPS throughout and not WIS at all.}

\item Some assessment of calibration, e.g. interval coverage, should be added.
\spencer{We can add this. I don't think it will look good, but maybe we recommend
further research involving calibration of posterior credible regions (Syring \& Martin 2019)
or generalized predictive distributions (Wu \& Martin 2021).}

\end{enumerate}

\item The authors should at least discuss the possibility of sharing certain parameters across geographic
units as done e.g. in
\href{https://www.nature.com/articles/s41467-021-23234-5}{https://www.nature.com/articles/s41467-021-23234-5}
by Osthus and Moran.

\end{enumerate}


\section*{Review BA2412-010R1AE1}

The present paper proposes a novel two component framework for projecting Health 
and Human Services (HHS) hospitalization, based on hospitalization and 
influenza-like illness (ILI) data. In a first, step ILI data are modelled which 
are then used as a predictive covariate in the modelling and projection of the 
hospitalisation data. Probabilistic forecasts are assessed in a simulation study 
using proper scoring rules.


\begin{enumerate}[1.]

\item Since this paper is submitted to Bayesian analysis, the authors should 
underline more the modelling aspects and why this is an important contribution 
for the Bayesian community. What is the impact and importance of priors. Does 
the model borrow information between different model parts, propagate 
uncertainty ... In the introduction none of this is mentioned so the question 
arises whether Bayesian Analysis is the right journal.

\spencer{The ILI models are hierarchical and intended to be made so that information
accross all previous seasons may be borrowed for modeling the current season.}

\item As I understand one challenge of forecasting hospitalisation data is that the 
start of the collection of these data is very recent, so that a model only 
based on these data has not much historic data to be based on. Thus, the idea 
of using ILI data is natural. The authors could emphasize more and discuss 
potential alternative models that could even be included in the simulation 
study. I am missing a comparison to models that are not the newly proposed 
ones within the paper.
\spencer{Should include comparison to a naive baseline (probably a random walk) 
and an out of the box ARMA or ARIMA, and maybe exponential smoothing.}

\item Why is it sensible to include ILI as a linear function when modelling 
hospitalisation. Please motivate more. 

\spencer{I think this is supported in figure 6 in the paper, though as mentioned 
in response to previous reviewer, maybe additional EDA would be helpful.}

\item The authors provide also state level predictions. However, I do not see 
where a spatial component is included in the model. Is this only via P or are 
all states just modelled independently. For influenza I would think that it 
might be reasonable to assume that neighbouring states show similar behaviour 
in both ILI and hosptialisation so that borrowing strength between states 
within one model seems sensible. Please discuss.

\end{enumerate}


\subsection*{Specific comments:}

\begin{enumerate}[-]

\item Section 2, page 3: please be more clear on what you mean by the 
"first seven season". Provide a time frame.

\item Figure 1: I was wondering whether the R-package geofacet could be useful to 
show also a spatial trend across US states, in addition to temporal trend, for 
certain years,  similar to the second plot provided here:

https://hafen.github.io/geofacet/

I could image that the curves, onset and general behaviour, varies across the 
states. The authors refer to this a bit in the discussion. In addition I would 
prefer to also have months names in the axis labels, to avoid confusion that 
week 1 is not the first week in January.

Can the hospitalization figures also be shown on state level at least in an appendix.

\item Section 3.5: Please be clear what the strength of a Bayesian inference scheme is. 

\item Equation 6): What does the tilde mean?
\spencer{The tilde indicates forecast ILI. Maybe it just shouldn't be there?}

\item Page 11: Please motivate the choice of an autoregressive process of order 1 
more clearly. Why not use a higher order? Would this be beneficial?
\spencer{We explored higher orders and found that they didn't provide any more
predictive power than order 1 did.}

\item Page 14: Please discuss the assumption that all components of $\alpha_s$ and 
$\sigma_\epsilon^2$ are assumed to be same across all seasons. Is this realistic?

\item Figure 7: Please make clear in the caption what the columns represent in an 
intuitive manner. 

\item Equation 11, 12, 13, 14: Please motivate. The authors write in Section 3.4 
that the modelling of discrepancy might require careful selection of priors. 
Please underline more how this is taken care of in this manuscript.

\spencer{It was precisely the inclusion of the discrepancy that caused the serious problems
with identifiability which motivated the modular Bayesian approach. }

\item Please check for typos.
  
\end{enumerate}
\end{document}